\chapter*{Conclusion}  % chapter* je necislovana kapitola
\addcontentsline{toc}{chapter}{Conclusion} % rucne pridanie do obsahu
\markboth{Conclusion}{Conclusion} % vyriesenie hlaviciek

In this work we focused on the problem of classifying pixels of the image into two categories,
namely driveable segments and non-driveable ones. This problem has been raised by the need for a
better vision module for the robot named Smelý Zajko.
Students annually participate at the competition
called RoboTour Outdoor Delivery Contest with this robot. The robot had been designed, built and
continuously improved in previous works. However, the vision module did not take that much
attention and we could not rely on predictions and information about driveable path.
The weather and diversity of the environment are the main causes why previous approaches
have not been successful enough. Based on recent advances with deep convolutional
neural networks applied to computer vision, we decided to replace the robot's vision module
with CNN models.

We made a survey of the best contemporary ConvNet architectures for semantic segmentation
and chose some of them to be tested on our robot. Training was conducted in supervised manner,
meaning we provided images along with their ground truth labels to the models.
There are two datasets of images and labels released along with this thesis.
The images in these datasets were taken prior to RoboTour contests in Lednice and Deggendorf.

Models trained on these images performed relatively well, so we chose ResNet to be used
in Deggendorf since it showed the best trade-off between accuracy and prediction speed.
The robot had some issues with localization and planning algorithm in
the beginning, but in the end we were able to fix it and the robot worked fine. As the only
one it drove beyond load point in that particular round. We obtained
13 points in this competition, which brought us second place overall. However, these models
were not fast enough in terms of prediction time and we tried to minimize them by reducing
the number of filters and removing some layers. This step allows us to run the robot
at higher speed. We compared these minimized models to the so called mobile models 
that have been proposed recently. Interestingly, small version of MobileNetV3 reached
almost 20 frames per second while our small ResNet topped at 10 frames per second,
which is also nice and quite efficient at the image resolution of $640\times 480$. 

Labeling entire dataset is often an expensive and time consuming process.
We used Active Learning to study whether we are able to train
the models with less labeled images while preserving reasonable accuracy. The models started
to train on a small subset of training images and queried another ones to be labeled after
each round.
This process was repeated until the stopping condition was met. We used three sampling
strategies, namely random, entropy and diversity sampling. Their detailed description is
presented in Chapter \ref{chapter:activelearning}. Diversity and entropy methods performed
better than random sampling and needed only half of the training dataset to reach an accuracy
comparable with that of the models trained on the full dataset.
Thus, we can conclude that it is worth considering to use active learning when
training the model.

In the future, we could study if it is possible to get rid of the labeling process completely
and train the model in a fully unsupervised way. It is currently a deeply studied topic and
the first works 
\cite{bib:chen2019unsupervised, bib:nguyen2019deepusps, bib:bielski2019emergence}
have been proposed recently at the NeurIPS conference.
We also bought ZED mini stereo camera capable of creating depth map, tracking the position and
giving us odometry information. It would be worth trying to incorporate the depth map
into prediction of driveable trail, which could further improve the accuracy.
